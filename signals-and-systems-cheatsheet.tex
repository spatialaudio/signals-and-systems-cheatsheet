\documentclass[landscape,columns=3]{cheatsheet} %https://ctan.org/pkg/cheatsheet?lang=en
\geometry{paper=a3paper}
\hypersetup{colorlinks = false, allbordercolors = white}
\title{Signals and Systems Cheat Sheet\\
\tiny{\url{https://github.com/spatialaudio/signals-and-systems-cheatsheet}}}
\author{
Robert Hauser,
\href{https://orcid.org/0000-0002-3010-0294}{Frank Schultz},
University of Rostock
}
\date{\today}
\usepackage{graphicx}
\usepackage{fouriernc}
\usepackage{amsmath}
\usepackage{amsfonts}
\usepackage{amssymb}
\usepackage{amsbsy}
\usepackage{bm}
\usepackage{trfsigns}
\usepackage{pgfplots}
\usepackage{caption}
\usepackage{trfsigns}
\usetikzlibrary{arrows.meta}

%macros
\usepackage{sig_sys_macros}

%constants
\def\T{2.5}
\def\Th{1.25}
\def\A{1.5}

%small commands
\newcommand{\sinc}{\mathrm{sinc}}
\newcommand{\rect}{\mathrm{rect}}
\begin{document}
%\maketitle

\section{Continuous-Time Domain}
%
\begin{itemize}
\setlength\itemsep{-0.5em}
\item time $t$ in s, (physical) frequency $f$ in Hz
%
\item angular frequency $\omega = 2 \pi f$ in rad/s
%
\item cycle duration $T_0$ in s
%
\item fundamental oscillation $f_0 = \frac{1}{T_0}$ in Hz
$\rightarrow \omega_0 = 2 \pi f_0 = \frac{2\pi}{T_0}$ in rad/s
%
\end{itemize}

%\includegraphics[width=2.925in]{test.png}

\subsection{Fundamental Signals}

\subsection{Laplace Transform}
%
\begin{align*}
X(s) = \int\limits_{-\infty}^{+\infty} x(t) \, \e^{-s t}\,\mathrm{d}t
\qquad
x(t) = \frac{1}{2\pi\im}\lim_{\omega\to\infty}\int\limits_{s=\sigma-\im\omega}^{\sigma+\im\omega} X(s) \,\e^{+s t}\,\mathrm{d}s
\end{align*}
\begin{minipage}{\columnwidth}
	\includegraphics[width=\columnwidth,keepaspectratio]{laplace}
\end{minipage}



\columnbreak
\subsection{Fourier Series (FS)}
%
%
%Rechteckschwingung
%
\begin{itemize}
\setlength\itemsep{-0.5em}
\item continuous-time, periodic signal $x(t)$
%
\item discrete-frequency, non-periodic spectrum $\tilde{X}[\mu]$
%
\item $t\in\mathbb{R}$ and $\mu,\nu\in\mathbb{Z}$
%
\item  $T_0$-periodicity: $x(t)=x(t+\nu T_0)$ with $\omega_0 = \frac{2\pi}{T_0}$
%
\end{itemize}
%
\begin{align*}
\tilde{X}[\mu] = \int\limits_{0}^{T_0} x(t) \, \e^{-\im \omega_0 \mu t}\,\mathrm{d}t
\qquad
x(t) = \frac{1}{T_0}\sum\limits_{\mu=-\infty}^{+\infty} \tilde{X}[\mu] \,\e^{+\im \omega_0 \mu t}
\end{align*}


\begin{minipage}{\columnwidth}
\includegraphics[width=\columnwidth,keepaspectratio]{fourier_series_rect}
%\captionof{figure}{zeitlicher Verlauf sowie Fourierkoeffizienten der Rechteckschwingung, ihrer Verschiebung sowie deren Faltung}
\end{minipage}
%
%
\begin{minipage}{\columnwidth}
\includegraphics[width=\columnwidth,keepaspectratio]{fourier_series_tri}
%\captionof{figure}{zeitlicher Verlauf sowie Fourierkoeffizienten der Dreieckschwingung, ihrer Verschiebung sowie deren Faltung}
\end{minipage}\\

%Fourierkoeffizienten Rechteckschwingung
%\begin{gather}
%	c_0=2\int_0^{\frac{T_h}{2}}A\mathrm{d}t 
%= 2At\Bigg |_{t=0}^{t=\frac{T_h}{2}}=A\cdot T_h\\
%	c_k=\int_{-\frac{T_h}{2}}^{\frac{T_h}{2}}A\e^{-\im k \omega_0 t}\mathrm{d}t=\frac{A}{\im k \omega_0}\left(\e^{\im \pi k \frac{T_h}{T}}-\e^{-\im \pi k \frac{T_h}{T}}\right)\\
%	=\frac{AT}{\im k 2\pi }\left(\e^{\im \pi k \frac{T_h}{T}}-\e^{-\im \pi k \frac{T_h}{T}}\right)
%	=\frac{AT}{k\pi}\sin(\pi k \frac{T_h}{T})\\
%	=\frac{AT}{k\pi}\frac{T_h}{T_h}\sin(\pi k \frac{T_h}{T})={AT_h}\frac{\sin(\pi k \frac{T_h}{T})}{\pi k \frac{T_h}{T}}=A\cdot T_h\mathrm{sinc}(\pi k \frac{T_h}{T})
%\end{gather}
%%%$x(t)\ast y(t)
%\begin{gather}
%	x(t)\ast y(t)=\begin{cases}
%		\int_{t-T_h}^{-T+\frac{T_h}{2}}\frac{A^2}{2}\mathrm{d}\tau=\frac{A^2}{2}\left(-T+\frac{T_h}{2}-t+T_h\right)=-\frac{A^2}{2}\left(t+T-3\frac{T_h}{2}\right)&,-\frac{T}{2}\leq t \leq -\frac{T_h}{2}\\
%		\int_{-\frac{T_h}{2}}^{t}\frac{A^2}{2}\mathrm{d}\tau=\frac{A^2}{2}\left(t+\frac{T_h}{2}\right)&,\frac{-T_h}{2}\leq t \leq \frac{T_h}{2} \\
%		\int_{t-T_h}^{\frac{T_h}{2}}\frac{A^2}{2}\mathrm{d}\tau=\frac{A^2}{2}\left(\frac{T_h}{2}-t+T_h\right)=-\frac{A^2}{2}\left(t-3\frac{T_h}{2}\right)&,\frac{T_h}{2}\leq t \leq \frac{T}{2}
%	\end{cases}
%\end{gather}

%Fourierkoeffizienten Dreieckschwingung

%\begin{gather}
%	c_k = \int_{-\frac{T_h}{2}}^{+\frac{T_h}{2}}x(t)\e^{-\im k \omega t}\mathrm{d}t=\int_{-\frac{T_h}{2}}^0\left(\frac{2A}{T_h}t+A\right)\e^{-\im k \omega t}\mathrm{d}t+\int_0^{+\frac{T_h}{2}}\left(-\frac{2A}{T_h}t+A\right)\e^{-\im k \omega t}\mathrm{d}t\\
%	=\frac{\left(\frac{2A}{T_h}t+A\right)}{-\im k \omega }\e^{-\im k \omega t}\Bigg|_{t=-\frac{T_h}{2}}^{0}+\int_{-\frac{T_h}{2}}^0\frac{\left(\frac{2A}{T_h}\right)}{\im k \omega }\e^{-\im k \omega t}\mathrm{d}t+\frac{\left(-\frac{2A}{T_h}t+A\right)}{-\im k \omega }\e^{-\im k \omega t}\Bigg|_{t=0}^{+\frac{T_h}{2}}-\int_0^{+\frac{T_h}{2}}\frac{\left(\frac{2A}{T_h}\right)}{\im k \omega }\e^{-\im k \omega t}\mathrm{d}t\\
%	=-\frac{A}{\im k \omega }+\frac{\left(\frac{2A}{T_h}\right)}{k^2\omega^2}\e^{-\im k \omega t}\Bigg |_{t=-\frac{T_h}{2}}^{t=0}+\frac{A}{\im k \omega}+\frac{\left(\frac{2A}{T_h}\right)}{k^2\omega^2}\e^{-\im k \omega t}\Bigg |_{t=+\frac{T_h}{2}}^{t=0}\\
%	=\frac{\left(\frac{2A}{T_h}\right)}{k^2\omega^2}\left[2-\e^{\im k \omega \frac{T_h}{2}}-\e^{-\im k \omega \frac{T_h}{2}}\right]=\frac{\left(\frac{4A}{T_h}\right)}{k^2\omega^2}\left[1-\cos(k\pi\frac{T_h}{T})\right]
%\end{gather}
%Es gilt (vgl. \cite[Kap. 2 S. 83]{Bronstein}):
%\begin{gather}
%	\sin^2(\alpha)=\frac{1}{2}(1-\cos(2\alpha))
%\end{gather}
%Also:
%\begin{gather}
%	\frac{\left(\frac{4A}{T_h}\right)}{k^2\omega^2}\left[1-\cos(k\pi\frac{T_h}{T})\right]=\frac{\left(\frac{4AT^2}{T_h}\right)}{k^24\pi^2}\left[1-\cos(k\pi\frac{T_h}{T})\right]=\frac{2AT^2}{T_hk^2\pi^2}\cdot\frac{1}{2}\left[1-\cos\left(2k\pi\frac{T_h}{2T}\right)\right]\\
%	=\frac{2AT^2}{T_hk^2\pi^2}\left[\sin^2\left(k\pi\frac{T_h}{2T}\right)\right]=\frac{T_hA}{2}\frac{4T^2}{T_h^2k^2\pi^2}\left[\sin^2\left(k\pi\frac{T_h}{2T}\right)\right]=\frac{A\cdot T_h}{2}\sinc^2\left(k\pi\frac{T_h}{2T}\right)
%\end{gather}

\columnbreak
\subsection{Fourier Transform (FT)}
%
%
\begin{itemize}
\setlength\itemsep{-0.5em}
\item continuous-time signal $x(t)$
%
\item continuous-frequency, non-periodic spectrum $X(\omega)$
%
\item $t, \omega\in\mathbb{R}$
%
\end{itemize}
%
\begin{align*}
X(\omega) = \int\limits_{-\infty}^{+\infty} x(t) \,\e^{-\im \omega t}\,\mathrm{d}t
\qquad
x(t) = \frac{1}{2\pi}\int\limits_{-\infty}^{+\infty} X(\omega) \,\e^{+\im \omega t}\,\mathrm{d}\omega
\end{align*}
%Fourier Trafo
\begin{minipage}{\columnwidth}
\includegraphics[width=\columnwidth,keepaspectratio]{continuous_fouriertrafo_rect}
\end{minipage}

\begin{minipage}{\columnwidth}
	\includegraphics[width=\columnwidth,keepaspectratio]{continuous_fouriertrafo_tri}
\end{minipage}




\columnbreak
%\hspace{2cm}
\section{Discrete-Time Domain}
%
\begin{itemize}
\setlength\itemsep{-0.5em}
%
\item time/sample index $k$
%
\item sampling interval $T_s$, sampling frequency $f_s = \frac{1}{T_s}$
%
\item temporal sampling: $\omega \cdot t \rightarrow \omega \cdot T_s k = \Omega k$
%
\item angular frequency $\Omega = \omega T_s = 2 \pi f T_s = 2 \pi \frac{f}{f_s}$
%
\end{itemize}


\subsection{Fundamental Signals}

\subsection{z-Transform}
%
\begin{align*}
X(z) = \sum\limits_{k=-\infty}^{+\infty} x[k] \,z^{-k}
\qquad
x[k] = \frac{1}{2\pi\im}\oint\limits_{C}^{} X(z) \,z^{k-1}\,\mathrm{d}z
\end{align*}
\begin{minipage}{\columnwidth}
	\includegraphics[width=\columnwidth,keepaspectratio]{zTrafo}
\end{minipage}
\columnbreak
\subsection{Discrete-Time Fourier Transform (DTFT)}
%
\begin{itemize}
\setlength\itemsep{-0.5em}
\item discrete-time signal $x[k]$
%
\item continuous-frequency, periodic spectrum $X(\Omega)$
%
\item $k,\nu\in\mathbb{Z}$ and $\Omega\in\mathbb{R}$
%
\item  $2\pi$-periodicity: $X(\Omega)=X(\Omega+\nu2\pi)$
\end{itemize}
%
\begin{align*}
X(\Omega) = \sum\limits_{k=-\infty}^{+\infty} x[k] \,\e^{-\im \Omega k}
\qquad
x[k] = \frac{1}{2\pi}\int\limits_{0}^{2\pi} X(\Omega) \,\e^{+\im \Omega k}\,\mathrm{d}\Omega
\end{align*}
%
%
\begin{minipage}{\columnwidth}
	\includegraphics[width=\columnwidth,keepaspectratio]{DTFT_rect.pdf}
\end{minipage}

%DTFT eines Dreieickes:
%Faltung von $\mathrm{rect}_[k]$ mit sich selbst ergibt ein Dreieck der Länge
%$2N+1$ mit Maximum bei N-1.
%%
%DTFT durch Multiplikation der DTFTs des Rechtecksignals:
%%
%\begin{gather}
%	\mathrm{rect}_N[k]\ast\mathrm{rect}_N[k]\quad\fourier\quad\e^{-\im\Omega(N-1)}\cdot\frac{\sin^2(N\frac{\Omega}{2})}{\sin^2(\frac{\Omega}{2})}
%\end{gather}
%%
%Mittelpunkt des Dreiecks soll bei 0 liegen, also Verschiebung um $N-1$ nach 
%links.
%%
%Durch Faltung steigt das Dreieck jeweils um 1 bis zu einem Maximum 
%bei $N-1$ von $N$.
%%
%\begin{gather}
%	N\cdot \mathrm{tri}_{2N+1}[k]=\e^{-\im\Omega(N-1)}\cdot\frac{\sin^2(N\frac{\Omega}{2})}{\sin^2(\frac{\Omega}{2})}\cdot\e^{+j\Omega(N-1)}=\frac{\sin^2(N\frac{\Omega}{2})}{\sin^2(\frac{\Omega}{2})}
%\end{gather}
%%
%N muss noch auf die andere Seite gebracht werden.
%%
%\begin{gather}
%	\mathrm{tri}_{2N+1}[k]\quad\fourier\quad\frac{1}{N}\cdot\frac{\sin^2(N\frac{\Omega}{2})}{\sin^2(\frac{\Omega}{2})}
%\end{gather}
%%
%Dies ist eine Korrespondenz für einen Dreiecksimpuls der Länge $2N+1$ mit der
%Spitz bei $k=0$.
%%
%\begin{gather}
%	\mathrm{tri}_N[k]=
%	\begin{cases}
%		1-\frac{k}{N}, &0 \leq k \leq N,\\
%		1+\frac{k}{N}, &-N\leq k < 0,\\
%		0, &\text{sonst}.
%	\end{cases}
%\end{gather}
%%
%%
%%
%\begin{gather}
%	\lim\limits_{\Omega\rightarrow0}\frac{1}{N}\cdot\frac{\sin^2(N\frac{\Omega}{2})}{\sin^2(\frac{\Omega}{2})}
%	%
%	=\lim\limits_{\Omega\rightarrow0}\frac{1}{N}\cdot\frac{2\sin\left(N\frac{\Omega}{2}\right)\cos\left(N\frac{\Omega}{2}\right)\frac{N}{2}}{2\sin\left(\frac{\Omega}{2}\right)\cos\left(\frac{\Omega}{2}\right)\frac{1}{2}}\\
%	%
%	=\lim\limits_{\Omega\rightarrow0}
%	\frac{
%		%Zaehlers
%		\cos^2\left(N\frac{\Omega}{2}\right)\frac{N}{2}-\sin^2\left(N\frac{\Omega}{2}\right)\frac{N}{2}
%	}
%	%Nenner
%	{
%		\cos^2\left(\frac{\Omega}{2}\right)\frac{1}{2}-\sin^2\left(\frac{\Omega}{2}\right)\frac{1}{2}
%	}
%	=N
%\end{gather}
%
%
%
\begin{minipage}{\columnwidth}
	\includegraphics[width=\columnwidth,keepaspectratio]{DTFT_tri.pdf}
\end{minipage}
%
\columnbreak
\subsection{Discrete Fourier Transform (DFT)}
%
\begin{itemize}
\setlength\itemsep{-0.5em}
\item discrete-time, periodic signal $x[k]$
%
\item discrete-frequency, periodic spectrum $X[\mu]$
%
\item $k,\mu,\nu\in\mathbb{Z}$
%
\item  $N$-periodicity: $x[k]=x[k + \nu N]$ and $X[\mu]=x[\mu + \nu N]$
\end{itemize}
%
\begin{align*}
X[\mu] = \sum\limits_{k=0}^{N-1} x[k] \,\e^{-\im \frac{2\pi}{N} \mu k}
\qquad
x[k] = \frac{1}{N}\sum\limits_{\mu=0}^{N-1} X[\mu] \,\e^{+\im \frac{2\pi}{N} \mu k}
\end{align*}

\begin{minipage}{\columnwidth}
	\includegraphics[width=\columnwidth,keepaspectratio]{DFT_rect.pdf}
\end{minipage}

\begin{minipage}{\columnwidth}
	\includegraphics[width=\columnwidth,keepaspectratio]{DFT_tri.pdf}
\end{minipage}
\columnbreak

\section{Signals and Systems Cheat Sheet}
\href{https://github.com/robhau}{Robert Hauser},
\href{https://orcid.org/0000-0002-3010-0294}{Frank Schultz},
University of Rostock\\
\tiny
\url{https://github.com/spatialaudio/signals-and-systems-cheatsheet}
\normalsize
\begin{description}
\item[v0.00 - 2021/xx/xx] TBD
\end{description}
\nocite{sigsysex,Bronstein}
\bibliography{literatur}
\bibliographystyle{ieeetran}
\end{document}
